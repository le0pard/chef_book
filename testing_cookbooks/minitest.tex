\section{Minitest}
\label{sec:testing-minitest}

\href{https://github.com/seattlerb/minitest}{Minitest} provides a complete suite of testing facilities supporting \href{http://en.wikipedia.org/wiki/Test-driven\_development}{TDD}, \href{http://en.wikipedia.org/wiki/Behavior-driven\_development}{BDD}, mocking, and benchmarking. Minitest doesn't reinvent anything that ruby already provides, like: classes, modules, inheritance, methods. This means you only have to learn ruby to use minitest and all of your regular OO practices like extract-method refactorings still apply.

Exists two way of usage minitest:

\begin{itemize}
  \item With Test Kitchen (like bats or serverspec)
  \item Testing node after cooking by minitest-handler
\end{itemize}

Let's conside each example.



\subsection{Test Kitchen}



\subsection{Minitest Handler}