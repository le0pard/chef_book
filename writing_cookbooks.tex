\chapter{Writing Cookbooks}

A cookbook is the fundamental unit of configuration and policy distribution. Each cookbook defines a scenario, such as everything needed to install and configure MySQL, and then it contains all of the components that are required to support that scenario.

As you read from previous chapter vendor cookbooks can help you to install and configure any possible software, but in most cases it is not enough. This is because you have you application, which need install, configure special cases only for this application. What is why you must to know how to write own Chef cookbooks.

Chef cookbooks is written on \href{https://www.ruby-lang.org}{Ruby} language. It is dynamic and open source programming language, which very well fit to use as \href{http://en.wikipedia.org/wiki/Domain-specific\_language}{DSL} for Chef recipes.

\section{Cookbook file organization}

For beginning we will generate cokbook by knife or berks. You can install both by bundler (we already did this in our kitchens). So, let's create <<my\_cool\_app>> cookbook:

\begin{lstlisting}[language=Bash,label=lst:cookbook-organization1]
$ cd site-cookbooks
$ knife cookbook create my_cool_app -o .
# or another way by berks
$ berks cookbook my_cool_app
\end{lstlisting}

After this you should see inside <<site-cookbooks>> new folder <<my\_cool\_app>>. This is our cookbook, which have such file structure inside:

\begin{lstlisting}[language=Bash,label=lst:cookbook-organization2]
$ ls -l my_cool_app
total 72
drwxr-xr-x  .
drwxr-xr-x  ..
drwxr-xr-x  .git
-rw-r--r--  .gitignore
-rw-r--r--  Berksfile
-rw-r--r--  Gemfile
-rw-r--r--@ LICENSE
-rw-r--r--  README.md
-rw-r--r--  Thorfile
-rw-r--r--  Vagrantfile
drwxr-xr-x  attributes
-rw-r--r--  chefignore
drwxr-xr-x  definitions
drwxr-xr-x  files
drwxr-xr-x  libraries
-rw-r--r--  metadata.rb
drwxr-xr-x  providers
drwxr-xr-x  recipes
drwxr-xr-x  resources
drwxr-xr-x  templates
\end{lstlisting}

Let's consider this structure:

\begin{itemize}
  \item \textbf{.git} - git repository skeleton (no need to do <<git init>>)
  \item \textbf{.gitignore} - specifies intentionally untracked files to ignore by Git
  \item \textbf{Berksfile} - file with cookbook dependencies for berkshelf (used for testing)
  \item \textbf{Gemfile} - file with gems for bundler (used for testing)
  \item \textbf{LICENSE} - file contain license information about cookbook
  \item \textbf{README.md} - file contains information about cookbook. <<.md>> mean \href{http://daringfireball.net/projects/markdown/syntax}{markdown} syntax
  \item \textbf{Thorfile} - file include tasks for thor gem (toolkit for building command-line interfaces, used for testing)
  \item \textbf{Vagrantfile} - file describe the type of machine required for a cookbook for vagrant
  \item \textbf{attributes} - ...
  \item \textbf{chefignore} - ...
  \item \textbf{definitions} - ...
  \item \textbf{files} - ...
  \item \textbf{attributes} - ...
  \item \textbf{metadata.rb} - ...
  \item \textbf{providers} - ...
  \item \textbf{recipes} - ...
  \item \textbf{resources} - ...
  \item \textbf{templates} - ...
\end{itemize}
\section{Metadata}
\label{sec:cookbook-metadata}

Metadata is file, which contain all main information about cookbook. Let's consider our generated example:

\begin{lstlisting}[label=lst:cookbook-metadata1,title=my-server-cloud/site-cookbooks/my\_cool\_app/metadata.rb]
name             'my_cool_app'
maintainer       'YOUR_NAME'
maintainer_email 'YOUR_EMAIL'
license          'All rights reserved'
description      'Installs/Configures my_cool_app'
long_description IO.read(File.join(File.dirname(__FILE__), 'README.md'))
version          '0.1.0'
\end{lstlisting}

This file written on Ruby and can have such settings:

\begin{itemize}
  \item \inline!name! - the name of the cookbook
  \item \inline!maintainer! - the name of the person responsible for maintaining a cookbook, either an individual or an organization
  \item \inline!maintainer_email! - the email address for the person responsible for maintaining a cookbook. Only one email can be listed here
  \item \inline!license! - the type of license under which a cookbook is distributed: <<Apache v2.0>>, <<GPL v2>>, <<GPL v3>>, <<MIT>>, or license <<Proprietary - All Rights Reserved>> (default)
  \item \inline!description! - a short description of a cookbook and its functionality
  \item \inline!long_description! - a longer description that ideally contains full instructions on the proper use of a cookbook, including definitions, libraries, dependencies, and so on. In example the contents pulled from <<README.md>> file
  \item \inline!version! - the current version of a cookbook. Version numbers always follow a simple three-number version sequence
  \item \inline!attribute! - the list of attributes that are required to configure a cookbook
  \item \inline!depends! - indicates that a cookbook has a dependency on another cookbook
  \item \inline!recommends! - adds a dependency on another cookbook that is recommended, but not required
  \item \inline!suggests! - adds a dependency on another cookbook that is suggested, but not required
  \item \inline!conflicts! - indicates that a cookbook conflicts with another cookbook or cookbook version
  \item \inline!grouping! - adds a title and description to a group of attributes within a namespace
  \item \inline!provides! - adds a recipe, definition, or resource that is provided by this cookbook, should the auto-populated list be insufficient
  \item \inline!recipe! - a description for a recipe, mostly for cosmetic value within the server user interface
  \item \inline!replaces! - indicates that this cookbook should replace another (and can be used in-place of that cookbook)
  \item \inline!supports! - indicates that a cookbook has a supported platform
\end{itemize}

We need modify it to set our information about this cookbook:

\begin{lstlisting}[label=lst:cookbook-metadata2,title=my-server-cloud/site-cookbooks/my\_cool\_app/metadata.rb]
name             'my_cool_app'
maintainer       'Alexey Vasiliev'
maintainer_email 'leopard_ne@inbox.ru'
license          'MIT'
description      'Installs/Configures my_cool_app'
long_description IO.read(File.join(File.dirname(__FILE__), 'README.md'))
version          '0.1.0'
\end{lstlisting}

As of writing this cookbook, we will be adding information to this file on it.

\section{Resources and Providers}

As you read from previous chapter, Chef inside have resources (in example we used <<package>> resource). A resource defines the actions that can be taken, such as when a package should be installed, whether a service should be enabled or restarted, which groups, users, or groups of users should be created, where to put a collection of files, what the name of a new directory should be, and so on. During a chef-client run, each resource is identified and then associated with a provider. The provider then does the work to complete the action defined by the resource. Each resource is processed in the same order as they appear in a recipe. The chef-client ensures that the same actions are taken the same way everywhere and that actions produce the same result every time. A resource is implemented within a recipe using Ruby.

Let's look at the most necessary resources.

\subsection{Bash}

The bash resource is used to execute scripts using the Bash interpreter and includes all of the actions and attributes that are available to the execute resource. Example:

\begin{lstlisting}[label=lst:cookbook-resources-bash]
bash "install_something" do
  user "root"
  cwd "/tmp"
  code <<-EOH
  wget http://www.example.com/tarball.tar.gz
  tar -zxf tarball.tar.gz
  cd tarball
  ./configure
  make
  make install
  EOH
end
\end{lstlisting}
\section{Recipes}

Any cookbook contains recipes. The default recipe inside cookbook have name <<default>>. Let's add our default recipe, which will install \href{http://git-scm.com/}{git}:

\begin{lstlisting}[label=lst:cookbook-recipes1,title=my-server-cloud/site-cookbooks/my\_cool\_app/recipes/default.rb]
#
# Cookbook Name:: my_cool_app
# Recipe:: default
#
# Copyright (C) 2014 Alexey Vasiliev
#
# MIT
#

package 'git'
\end{lstlisting}

As you can see, at the beginning of recipe we have comments about this recipe. Next we add resource <<package>> with argument <<git>>. The <<package>> resource is used to manage packages on the system. For example, on Debian or Ubuntu resource <<package>> will use <<apt-get>> command to install git on system.

Now you should add <<my\_cool\_app>> into run-list to use this cookbook:

\begin{lstlisting}[label=lst:cookbook-recipes2,title=my-server-cloud/site-cookbooks/nodes/second.example.com.json]
{
  "name": "second.example.com",
  "json_class": "Chef::Node",
  "chef_type": "node",
  "chef_environment": "development",
  "normal": {
    "fqdn": "10.33.33.35"
  },
  "default": {},
  "override": {},
  "run_list": [
    "role[chef-client]",
    "role[nginx]",
    "my_cool_app"
  ]
}
\end{lstlisting}

If you using Chef Server, don't forget upload this cookbook and update node on Chef Server by knife.

\begin{lstlisting}[language=Bash,label=lst:cookbook-recipes3]
$ knife cookbook upload my_cool_app
Uploading my_cool_app    [0.1.0]
Uploaded 1 cookbook.
$ knife node from file nodes/second.example.com.json
Updated Node second.example.com!
// on real environment you will execute "knife ssh 'name:second.example.com' 'sudo chef-client' -i ../keys/production.pem -x ubuntu"
$ vagrant provision chef_second_client
[2014-01-21T18:21:55+00:00] INFO: Chef Run complete in 26.935610739 seconds
[2014-01-21T18:21:55+00:00] INFO: Running report handler
\end{lstlisting}

Let's install also \href{http://en.wikipedia.org/wiki/Network\_Time\_Protocol}{ntp} package in the same recipe. Because we have in recipe Ruby syntax, we can little \href{http://ru.wikipedia.org/wiki/Dont\_repeat\_yourself}{DRY} our code:

\begin{lstlisting}[label=lst:cookbook-recipes4,title=my-server-cloud/site-cookbooks/my\_cool\_app/recipes/default.rb]
%w(git ntp).each do |pack|
  package pack
end
\end{lstlisting}

Again upload cookbook and run chef-client:

\begin{lstlisting}[language=Bash,label=lst:cookbook-recipes5]
$ knife cookbook upload my_cool_app
Uploading my_cool_app    [0.1.0]
Uploaded 1 cookbook.
// on real environment you will execute "knife ssh 'name:second.example.com' 'sudo chef-client' -i ../keys/production.pem -x ubuntu"
$ vagrant provision chef_second_client
[2014-01-21T18:21:55+00:00] INFO: Chef Run complete in 26.935610739 seconds
[2014-01-21T18:21:55+00:00] INFO: Running report handler
$ vagrant ssh chef_second_client
...
vagrant@precise64:~$ ps ax | grep ntp
 1115 ?        Ss     0:00 /usr/sbin/ntpd -p /var/run/ntpd.pid -g -u 103:108
13839 pts/2    S+     0:00 grep --color=auto ntp
vagrant@precise64:~$ git --version
git version 1.7.9.5
\end{lstlisting}

As you can see our simple cookbook is working.
\section{Attributes}

An attribute can be defined in a cookbook (or a recipe) and then used to override the default settings on a node. When a cookbook is loaded during a chef-client run, these attributes are compared to the attributes that are already present on the node. When the cookbook attributes take precedence over the default attributes, the chef-client will apply those new settings and values during the chef-client run on the node.

An attribute file is located in the <<attributes/>> sub-directory for a cookbook. When a cookbook is run against a node, the attributes contained in all attribute files are evaluated in the context of the node object. Node methods (when present) are used to set attribute values on a node. For example, the apache2 cookbook contains an attribute file called <<default.rb>>, which contains the following attributes:

\begin{lstlisting}[label=lst:cookbook-attributes1]
default["apache"]["dir"]          = "/etc/apache2"
default["apache"]["listen_ports"] = [ "80","443" ]
\end{lstlisting}

The use of the node object (node) is implicit in the previous example; the following example defines the node object itself as part of the attribute:

\begin{lstlisting}[label=lst:cookbook-attributes2]
node.default["apache"]["dir"]          = "/etc/apache2"
node.default["apache"]["listen_ports"] = [ "80","443" ]
\end{lstlisting}

\section{Templates}

As you can see in previous chapter we used resource \inline{template} for generate nginx config in default recipe. A cookbook template is a file written in a markup language that allows the contents of a file to be dynamically generated based on variables or complex logic. Templates can contain Ruby expressions and statements. Templates are a great way to manage configuration files across an organization. A template requires a template resource being added to a recipe and then a corresponding Embedded Ruby (ERB) template being added to a cookbook.

To use a template, two things must happen:

\begin{itemize}
  \item A template resource must be added to a recipe
  \item An Embedded Ruby (ERB) template must be added to a cookbook
\end{itemize}

For example, the following template file and template resource settings can be used to manage a configuration file named \inline{/etc/sudoers}. Within a cookbook that uses sudo, the following resource could be added to \inline{recipes/default.rb}:

\begin{lstlisting}[label=lst:cookbook-templates1]
template "/etc/sudoers" do
  source "sudoers.erb"
  mode 0440
  owner "root"
  group "root"
  variables({
     :sudoers_groups => node[:authorization][:sudo][:groups],
     :sudoers_users => node[:authorization][:sudo][:users]
  })
end
\end{lstlisting}

And then create a template called \inline{sudoers.erb} and save it to \inline{templates/default/sudoers.erb}:

\begin{lstlisting}[label=lst:cookbook-templates2]
#
# /etc/sudoers
#
# Generated by Chef for <%= node[:fqdn] %>
#

Defaults        !lecture,tty_tickets,!fqdn

# User privilege specification
root          ALL=(ALL) ALL

<% @sudoers_users.each do |user| -%>
<%= user %>   ALL=(ALL) <%= "NOPASSWD:" if @passwordless %>ALL
<% end -%>

# Members of the sysadmin group may gain root privileges
%sysadmin     ALL=(ALL) <%= "NOPASSWD:" if @passwordless %>ALL

<% @sudoers_groups.each do |group| -%>
# Members of the group '<%= group %>' may gain root privileges
%<%= group %> ALL=(ALL) <%= "NOPASSWD:" if @passwordless %>ALL
<% end -%>
\end{lstlisting}

And then set the default attributes in \inline{attributes/default.rb}:

\begin{lstlisting}[label=lst:cookbook-templates3]
default["authorization"]["sudo"]["groups"] = [ "sysadmin","wheel","admin" ]
default["authorization"]["sudo"]["users"]  = [ "jerry","greg"]
\end{lstlisting}

When a template is rendered, Ruby expressions and statements are evaluated by the chef-client. The variables listed in the resource’s variables parameter and the node object are evaluated. The chef-client then passes these variables to the template, where they will be accessible as instance variables within the template; the node object can be accessed just as if it were part of a recipe, using the same syntax.

For example, a simple template resource like this:

\begin{lstlisting}[label=lst:cookbook-templates4]
node[:fqdn] = "latte"
template "/tmp/foo" do
  source "foo.erb"
  variables({
    :x_men => "are keen"
  })
end
\end{lstlisting}

And a simple Embedded Ruby (ERB) template like this:

\begin{lstlisting}[label=lst:cookbook-templates5]
The node <%= node[:fqdn] %> thinks the x-men <%= @x_men %>
\end{lstlisting}

Would render something like:

\begin{lstlisting}[label=lst:cookbook-templates6]
The node latte thinks the x-men are keen
\end{lstlisting}

Even though this is a very simple example, the full capabilities of Ruby can be used to tackle even the most complex and demanding template requirements.

\subsection{File Specificity}

A cookbook will frequently be designed to work across many platforms and will often be required to distribute a specific file to a specific platform. A cookbook can be designed to support distributing files across platforms, but ensuring that the right file ends up on each system.

TODO
\section{LWRPs}
\label{sec:cookbook-lwrp}

A LWRP (Lightweight Resources and Providers) is a part of a cookbook that is used to extend the chef-client in a way that allows custom actions to be defined, and then used in recipes in much the same way as any platform resource. A LWRP has two principal components:

\begin{itemize}
  \item A lightweight resource that defines a set of actions and attributes
  \item A lightweight provider that tells the chef-client how to handle each action, what to do if certain conditions are met, and so on
\end{itemize}

In addition, most lightweight providers are built using platform resources and some lightweight providers are built using custom Ruby code.

Once created, a LWRP becomes a Ruby class within the organization. During each chef-client run, the chef-client will read the lightweight resources from recipes and process them alongside all of the other resources. When it is time to configure the node, the chef-client will use the corresponding lightweight provider to determine the steps required to bring the system into the desired state.

Where the lightweight resource represents a piece of the system, its current state, and the action that is needed to move it to the desired state, a lightweight provider defines the steps that are required to bring that piece of the system from its current state to the desired state. A LWRP behaves similar to platform resources and providers:

\begin{itemize}
  \item A lightweight resource is a key part of a recipe
  \item A lightweight resource defines the actions that can be taken
  \item During a chef-client run, each lightweight resource is identified, and then associated with a lightweight provider
  \item A lightweight provider does the work to complete the action requested by the lightweight resource
\end{itemize}

Lightweight resources and providers are loaded from files that are saved in the following cookbook sub-directories:

\begin{tabular}{ | l | l | }
  \hline
  Directory	& Description \\
  \hline
  providers/ & The sub-directory in which lightweight providers are located. \\
  resources/ & The sub-directory in which lightweight resources are located. \\
  \hline
\end{tabular}

The naming patterns of lightweight resources and providers are determined by the name of the cookbook and by the name of the files in the <<resources/>> and <<providers/>> sub-directories. For example, if a cookbook named <<example>> was downloaded to the chef-repo, it would be located at <</cookbooks/example/>>. If that cookbook contained two resources and two providers, the following files would be part of the <<resources/>> directory:

\begin{tabular}{ | l | l | l | }
  \hline
  Files	& Resource Name	& Generated Class \\
  \hline
  default.rb & example & Chef::Resource::Example \\
  custom.rb	& example\_custom & Chef::Resource::ExampleCustom \\
  \hline
\end{tabular}

And the following files would be part of the <<providers/>> directory:

\begin{tabular}{ | l | l | l | }
  \hline
  Files	& Provider Name	& Generated Class \\
  \hline
  default.rb & example & Chef::Provider::Example \\
  custom.rb	& custom & Chef::Provider::ExampleCustom \\
  \hline
\end{tabular}

Let's add in our <<my\_cool\_app>> LWRP, which will add in \inline!/etc/ssh/ssh_known_hosts! host.

\subsection{Resources}

First of all we should create direcotry <<resources>> and add to it \inline!know_host.rb! file with content:

\begin{lstlisting}[label=lst:cookbook-lwrp1,title=my-server-cloud/site-cookbooks/my\_cool\_app/resources/know\_host.rb]
actions :create, :delete
default_action :create

attribute :host, :kind_of => String, :name_attribute => true, :required => true
attribute :key, :kind_of => String
attribute :port, :kind_of => Fixnum, :default => 22
attribute :known_hosts_file, :kind_of => String, :default => '/etc/ssh/ssh_known_hosts'

# Needed for Chef versions < 0.10.10
def initialize(*args)
  super
  @action = :create
end
\end{lstlisting}

Let's take it line by line. The first line specifies the allowed actions. Actions are what your resource can do, e.g. start, stop, create, delete, etc. In this case, you can \inline!:create! or \inline!:delete! known host. The next line defines the \inline!default_action! for our resource, in this case \inline!:create!. If you don't specify an action when you use the resource in a recipe, it will default to creating a known host, which is what you probably want. A general philosophy of Chef is to define intelligent or <<sane>> defaults.

Lines 4-7 define attributes, or properties of the known host resource we are creating. Line 4 defines an \inline!:host! attribute. Its \inline!:name_attribute! is true, which means that this attribute will be set to the string between \inline!my_cool_app_know_host! and \inline!do!. Example:

\begin{lstlisting}[label=lst:cookbook-lwrp2]
my_cool_app_know_host "Add github host" do
  host 'github.com'
end

my_cool_app_know_host 'github.com' do
  # The :host attribute will be set to 'github.com'
end
\end{lstlisting}

In the second example above, the \inline!:host! attribute wll be set to <<github.com>>.

Also, on line 4, we are definining the \inline!kind_of! validation parameter to tell the resource which kind of data we should expect (in this case, a string), whether this attribute is required (yes). Line 5 defines a \inline!:key! attribute, which is an optional string with no default. Line 6 defines a \inline!:port! attribute, a Ruby Fixnum (i.e. an integer) with a default of 22, which is the default when you create a known host. Line 7 defines a \inline!:known_hosts_file! attribute, a string with a default of \inline!/etc/ssh/ssh_known_hosts!, which is the default file with known hosts for ssh client.

For example, the \inline!cron_d! lightweight resource (found in the cron cookbook) can be used to manage files located in \inline!/etc/cron.d!:

\begin{lstlisting}[label=lst:cookbook-lwrp3]
actions :create, :delete
default_action :create

attribute :name, :kind_of => String, :name_attribute => true
attribute :cookbook, :kind_of => String, :default => "cron"
attribute :minute, :kind_of => [Integer, String], :default => "*"
attribute :hour, :kind_of => [Integer, String], :default => "*"
attribute :day, :kind_of => [Integer, String], :default => "*"
attribute :month, :kind_of => [Integer, String], :default => "*"
attribute :weekday, :kind_of => [Integer, String], :default => "*"
attribute :command, :kind_of => String, :required => true
attribute :user, :kind_of => String, :default => "root"
attribute :mailto, :kind_of => [String, NilClass]
attribute :path, :kind_of => [String, NilClass]
attribute :home, :kind_of => [String, NilClass]
attribute :shell, :kind_of => [String, NilClass]
\end{lstlisting}

where

\begin{itemize}
  \item the \inline!actions! allow a recipe to manage entries in a crontab file (create entry, delete entry)
  \item \inline!:create! is the default action
  \item \inline!:minute!, \inline!:hour!, \inline!:day!, \inline!:month!, and \inline!:weekday! are the collection of attributes used to schedule a cron job, assigned a default value of <<*>>
  \item \inline!:command! is the command that will be run (and also required)
  \item \inline!:user! is the user by which the command is run
  \item \inline!:mailto!, \inline!:path!, \inline!:home!, and \inline!:shell! are optional environment variables that do not have default value, which each being defined as an array that supports the String and NilClass Ruby classes
\end{itemize}

\subsection{Providers}

Now we need to create file \inline!know_host.rb! in <<providers>> directory:

\begin{lstlisting}[label=lst:cookbook-lwrp4,title=my-server-cloud/site-cookbooks/my\_cool\_app/providers/know\_host.rb]
# Support whyrun
def whyrun_supported?
  true
end

action :create do
  key, comment = insure_for_file(new_resource)
  # Use a Ruby block to edit the file
  ruby_block "add #{new_resource.host} to #{new_resource.known_hosts_file}" do
    block do
      file = ::Chef::Util::FileEdit.new(new_resource.known_hosts_file)
      file.insert_line_if_no_match(/#{Regexp.escape(comment)}|#{Regexp.escape(key)}/, key)
      file.write_file
    end
  end
  new_resource.updated_by_last_action(true)
end

action :delete do
  key, comment = insure_for_file(new_resource)
  # Use a Ruby block to edit the file
  ruby_block "del #{new_resource.host} from #{new_resource.known_hosts_file}" do
    block do
      file = ::Chef::Util::FileEdit.new(new_resource.known_hosts_file)
      file.search_file_delete_line(/#{Regexp.escape(comment)}|#{Regexp.escape(key)}/)
      file.write_file
    end
  end
  new_resource.updated_by_last_action(true)
end

def insure_for_file(new_resource)
  key = (new_resource.key || `ssh-keyscan -H -p #{new_resource.port} #{new_resource.host} 2>&1`)
  comment = key.split("\n").first || ""

  Chef::Application.fatal! "Could not resolve #{new_resource.host}" if key =~ /getaddrinfo/

  # Ensure that the file exists and has minimal content (required by Chef::Util::FileEdit)
  file new_resource.known_hosts_file do
    action        :create
    backup        false
    content       '# This file must contain at least one line. This is that line.'
    only_if do
      !::File.exists?(new_resource.known_hosts_file) || ::File.new(new_resource.known_hosts_file).readlines.length == 0
    end
  end
  [key, comment]
end
\end{lstlisting}

\subsection{DSL Methods}

\subsubsection{action}

The action method is used to define the steps that will be taken for each of the possible actions defined by the lightweight resource. Each action must be defined in separate action blocks within the same file. The syntax for the action method is as follows:

\begin{lstlisting}[label=lst:cookbook-lwrp-dsl-action]
action :action_name do
  if @current_resource.exists
    Chef::Log.info "#{ @new_resource } already exists - nothing to do."
  else
    resource "resource_name" do
      Chef::Log.info "#{ @new_resource } created."
    end
  end
  new_resource.updated_by_last_action(true)
end
\end{lstlisting}

where:

\begin{itemize}
  \item \inline!:action_name! corresponds to an action defined by a lightweight resource
  \item if \inline!@current_resource.exists! is a condition test that is using an instance variable to see if the object already exists on the node; this is an example of a test for idempotence
  \item If the object already exists, a \inline!@new_resource! already exists - nothing to do. log entry is created
  \item If the object does not already exists, the resource block is run. This block is a recipe that tells the chef-client what to do. A \inline!@new_resource! created. log entry is created
\end{itemize}

\subsubsection{current\_resource}

The \inline!current_resource! method is used to represent a resource as it exists on the node at the beginning of the chef-client run. In other words: what the resource is currently. The chef-client compares the resource as it exists on the node to the resource that is created during the chef-client run to determine what steps need to be taken to bring the resource into the desired state. This method is often used as an instance variable (\inline!@current_resource!).

For example:

\begin{lstlisting}[label=lst:cookbook-lwrp-dsl-current-resource]
action :add do
  unless @current_resource.exists
    cmd = "#{appcmd} add app /site.name:\"#{@new_resource.app_name}\""
    cmd << " /path:\"#{@new_resource.path}\""
    cmd << " /applicationPool:\"#{@new_resource.application_pool}\"" if @new_resource.application_pool
    cmd << " /physicalPath:\"#{@new_resource.physical_path}\"" if @new_resource.physical_path
    Chef::Log.debug(cmd)
    shell_out!(cmd)
    Chef::Log.info("App created")
  else
    Chef::Log.debug("#{@new_resource} app already exists - nothing to do")
  end
end
\end{lstlisting}

where the \inline!unless! conditional statement checks to make sure the resource doesn't already exist on a node, and then runs a series of commands when it doesn't. If the resource already exists, the log entry would be <<Foo app already exists - nothing to do.>>

\subsubsection{load\_current\_resource}

The \inline!load_current_resource! method is used to find a resource on a node based on a collection of attributes. These attributes are defined in a lightweight resource and are loaded by the chef-client when processing a recipe during a chef-client run. This method will ask the chef-client to look on the node to see if a resource exists with specific matching attributes.

For example:

\begin{lstlisting}[label=lst:cookbook-lwrp-dsl-load-current-resource]
def load_current_resource
  @current_resource = Chef::Resource::TransmissionTorrentFile.new(@new_resource.name)
  Chef::Log.debug("#{@new_resource} torrent hash = #{torrent_hash}")
  path = "foo:#{@new_resource.att1}@#{@new_resource.att2}:#{@new_resource.att3}/path"
  @transmission = Opscode::Transmission::Client.new(path)
  @torrent = nil
  begin
    @torrent = @transmission.get_torrent(torrent_hash)
    info = "Found existing #{@new_resource} in swarm "+
    "with name of '#{@torrent.name}' and status of '#{@torrent.status_message}'"
    Chef::Log.info(info)
    @current_resource.torrent(@new_resource.torrent)
  rescue
    Chef::Log.debug("Cannot find #{@new_resource} in the swarm")
  end
  @current_resource
end
\end{lstlisting}

In the previous example, if a resource exists with matching attributes, the chef-client does nothing and if a resource does not exist with matching attributes, the chef-client will enforce the state declared in \inline!new_resource!.

\subsubsection{new\_resource}

The \inline!new_resource! method is used to represent a resource as loaded by the chef-client during the chef-client run. In other words: what the resource should be. The chef-client compares the resource as it exists on the node to the resource that is created during the chef-client run to determine what steps need to be taken to bring the resource into the desired state.

For example:

\begin{lstlisting}[label=lst:cookbook-lwrp-dsl-new-resource]
action :delete do
  if exists?
    if ::File.writable?(new_resource.path)
      Chef::Log.info("Deleting #{new_resource} at #{new_resource.path}")
      ::File.delete(new_resource.path)
      new_resource.updated_by_last_action(true)
    else
      raise "Cannot delete #{new_resource} at #{new_resource.path}!"
    end
  end
end
\end{lstlisting}

where the chef-client checks to see if the file exists, then if the file is writable, and then attempts to delete the resource. \inline!path! is an attribute of the new resource that is defined by the lightweight resource.

\subsubsection{updated\_by\_last\_action}

The \inline!updated_by_last_action! method is used to notify a lightweight resource that a node was updated successfully. For example, the \inline!cron_d! lightweight resource in the cron cookbook:

\begin{lstlisting}[label=lst:cookbook-lwrp-dsl-updated]
action :create do
  t = template "/etc/cron.d/#{new_resource.name}" do
    cookbook new_resource.cookbook
    source "cron.d.erb"
    mode "0644"
    variables({
        :name => new_resource.name,
        :minute => new_resource.minute,
        :hour => new_resource.hour,
        :day => new_resource.day,
        :month => new_resource.month,
        :weekday => new_resource.weekday,
        :command => new_resource.command,
        :user => new_resource.user,
        :mailto => new_resource.mailto,
        :path => new_resource.path,
        :home => new_resource.home,
        :shell => new_resource.shell
      })
    action :create
  end
  new_resource.updated_by_last_action(t.updated_by_last_action?)
end
\end{lstlisting}

where \inline!t.updated_by_last_action?! uses a variable to check whether a new crontab entry was created.

\subsubsection{use\_inline\_resources}

A lightweight resource should be set to inline compile mode by adding the \inline!use_inline_resources! method at the top of the provider. This ensures that notifications work properly across the resource collection. The \inline!use_inline_resources! method was added to the chef-client starting in version 11.0 to address the behavior described below. The \inline!use_inline_resources! method should be considered a requirement for any lightweight resource authored against the 11.0+ versions of the chef-client. This behavior will become the default behavior in an upcoming version of the chef-client.

\subsubsection{whyrun\_supported?}

why-run mode is a way to see what the chef-client would have configured, had an actual chef-client run occurred. This approach is similar to the concept of <<no-operation>> (or <<no-op>>): decide what should be done, but then don't actually do anything until it's done right. This approach to configuration management can help identify where complexity exists in the system, where inter-dependencies may be located, and to verify that everything will be configured in the desired manner.

When why-run mode is enabled, a chef-client run will occur that does everything up to the point at which configuration would normally occur. This includes getting the configuration data, authenticating to the server, rebuilding the node object, expanding the run list, getting the necessary cookbook files, resetting node attributes, identifying the resources, and building the resource collection and does not include mapping each resource to a provider or configuring any part of the system.

When the chef-client is run in why-run mode, certain assumptions are made:

\begin{itemize}
  \item If the service resource cannot find the appropriate command to verify the status of a service, why-run mode will assume that the command would have been installed by a previous resource and that the service would not be running
  \item For \inline!not_if! and \inline!only_if! attribute, why-run mode will assume these are commands or blocks that are safe to run. These conditions are not designed to be used to change the state of the system, but rather to help facilitate idempotency for the resource itself. That said, it may be possible that these attributes are being used in a way that modifies the system state
  \item The closer the current state of the system is to the desired state, the more useful why-run mode will be. For example, if a full run-list is run against a fresh system, that run-list may not be completely correct on the first try, but also that run-list will produce more output than smaller run-list
\end{itemize}

The \inline!whyrun_supported?! method is used to set a lightweight provider to support why-run mode. The syntax for the \inline!whyrun_supported?! method is as follows:

\begin{lstlisting}[label=lst:cookbook-lwrp-why-run]
def whyrun_supported?
  true
end
\end{lstlisting}

where \inline!whyrun_supported?! is set to true for any lightweight provider that supports using why-run mode. When why-run mode is supported by the a lightweight provider, the \inline!converge_by! method is used to define the strings that are logged by the chef-client when it is run in why-run mode.

\subsubsection{Log entries and rescue}

Use the \inline!Chef::Log! class in a lightweight provider to define log entries that are created during a chef-client run. The syntax for a log message is as follows:

\begin{lstlisting}[label=lst:cookbook-lwrp-logs1]
Chef::Log.log_type("message")
\end{lstlisting}

where

\begin{itemize}
  \item \inline!log_type! can be .debug, .info, .warn, .error, or .fatal
  \item <<message>> is the message that is logged
\end{itemize}

For example, from the <<repository.rb>> provider in the yum cookbook:

\begin{lstlisting}[label=lst:cookbook-lwrp-logs2]
action :add do
  unless ::File.exists?("/etc/yum.repos.d/#{new_resource.repo_name}.repo")
    Chef::Log.info "Adding #{new_resource.repo_name} repository to /etc/yum.repos.d/#{new_resource.repo_name}.repo"
    repo_config
  end
end
\end{lstlisting}

where the \inline!Chef::Log! class appends .info as the log type. If the name of the repo was <<foo>>, then the log message would be <<Adding foo repository to /etc/yum.repos.d/foo.repo>>.

Another example shows two log entries, one that is triggered when a service is being restarted, and then another that is triggered after the service has been restarted:

\begin{lstlisting}[label=lst:cookbook-lwrp-logs3]
action :restart do
  if @current_resource.running
    Chef::Log.debug "Restarting #{new_resource.service_name}"
    shell_out!(restart_command)
    new_resource.updated_by_last_action(true)
    Chef::Log.debug "Restarted #{new_resource.service_name}"
  end
end
\end{lstlisting}

Use the \inline!rescue! clause to make sure that a log message is always provided. For example:

\begin{lstlisting}[label=lst:cookbook-lwrp-logs4]
def load_current_resource
  ...
  begin
    ...
  rescue
    Chef::Log.debug("Cannot find #{@new_resource} in the swarm")
  end
  ...
end
\end{lstlisting}

\subsection{Using LWRP}

We finished with our LWRP. Let's test it. Just add to <<default.rb>> you new LWRP, which will add to known hosts <<github.com>>:

\begin{lstlisting}[label=lst:cookbook-lwrp5,title=my-server-cloud/site-cookbooks/my\_cool\_app/recipes/default.rb]
# known hosts for github.com
my_cool_app_know_host 'github.com'
\end{lstlisting}

Upload cookbook on server and run chef-client on node:

\begin{lstlisting}[language=Bash,label=lst:cookbook-lwrp6]
$ knife cookbook upload my_cool_app
  Uploading my_cool_app    [0.1.0]
  Uploaded 1 cookbook.

$ vagrant provision chef_second_client
...
INFO: Found chef-client in /usr/bin/chef-client
INFO: runit_service[nginx] configured
INFO: ruby_block[add github.com to /etc/ssh/ssh_known_hosts] called
INFO: Chef Run complete in 11.920336698 seconds
INFO: Running report handlers
INFO: Report handlers complete

$ vagrant ssh chef_second_client

$ cat /etc/ssh/ssh_known_hosts
# This file must contain at least one line. This is that line.
# github.com SSH-2.0-OpenSSH_5.9p1 Debian-5ubuntu1+github5
|1|fvGKZG+jIkEntM5yBvzJ230TX1o=|9qP2wRdFIS+cAouirLYDb1Ibl7A= ssh-rsa...
\end{lstlisting}

Let's check how it will delete this hosts:

\begin{lstlisting}[label=lst:cookbook-lwrp7,title=my-server-cloud/site-cookbooks/my\_cool\_app/recipes/default.rb]
my_cool_app_know_host 'github.com' do
  action :delete
end
\end{lstlisting}

Again upload on server and run chef-client:

\begin{lstlisting}[language=Bash,label=lst:cookbook-lwrp8]
$ knife cookbook upload my_cool_app
  Uploading my_cool_app    [0.1.0]
  Uploaded 1 cookbook.

$ vagrant provision chef_second_client
...
INFO: Found chef-client in /usr/bin/chef-client
INFO: runit_service[nginx] configured
INFO: ruby_block[del github.com from /etc/ssh/ssh_known_hosts] called
INFO: Chef Run complete in 17.109379291 seconds
INFO: Running report handlers
INFO: Report handlers complete

$ vagrant ssh chef_second_client

$ cat /etc/ssh/ssh_known_hosts
# This file must contain at least one line. This is that line.
\end{lstlisting}

As you can see, all works as expected.


\section{Summary}

TODO