\section{Installation}

Exists several ways to install own Chef Server:

\begin{itemize}
  \item Go to \href{http://www.getchef.com/chef/install/}{www.getchef.com/chef/install}, select the operating system, version, and architecture of the server and install the downloaded package on it. After installation you can configure server by command <<sudo chef-server-ctl reconfigure>>
  \item Use Chef Solo to install Chef Server
\end{itemize}

Of course, I prefer to use Chef Solo to install and configure Chef Server. Chef Solo will help us quickly deploy Chef Server on a new server, if with it something happens (crash file system of server, etc.). Do not forget to make a backups of Chef Server (because compared with Chef Solo, Chef Server will be the point of failure in your configuration management system).

Let's create our folder, which will contain all our Chef kitchen:

\begin{lstlisting}[language=Bash,label=lst:my-server-cloud-installation1]
$ mkdir my-server-cloud
$ cd my-server-cloud
$ cat Gemfile
source "https://rubygems.org"

gem 'knife-solo'
gem 'berkshelf'
$ bundle
$ knife solo init .
WARNING: No knife configuration file found
Creating kitchen...
Creating knife.rb in kitchen...
Creating cupboards...
Setting up Berkshelf...
\end{lstlisting}

To install and configure Chef Server exists cookbook \href{http://community.opscode.com/cookbooks/chef-server}{chef-server}. Let's add this cookbook in Berkshelf:

\begin{lstlisting}[label=lst:my-server-cloud-installation2,title=my-server-cloud/Berkshelf]
site :opscode

cookbook 'chef-server'
\end{lstlisting}

After running the command <<berks install>> this cookbook will be installed with dependencies.

\begin{lstlisting}[language=Bash,label=lst:my-server-cloud-installation3]
$ berks install
Installing chef-server (2.0.1) from site: 'http://cookbooks.opscode.com/api/v1/cookbooks'
\end{lstlisting}

Now we should configure a Chef Solo node for our Chef Server.