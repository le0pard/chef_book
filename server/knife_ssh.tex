\section{Knife ssh}

The knife ssh subcommand is used to invoke SSH commands (in parallel) on a subset of nodes within an organization, based on the results of a search query. We already use it to run chef-client on <<first.example.com>> node. Let's consider an examples.

To find the uptime of all of web servers (all node, which have role <<web>>):

\begin{lstlisting}[language=Bash,label=lst:my-server-cloud-knife-ssh1]
$ knife ssh "role:web" "uptime" -i ../keys/production.pem -x ubuntu
***.com    13:18:28 up 55 days, 14 min,  1 user,  load average: 0.00, 0.01, 0.05
***.com    13:18:28 up 75 days, 23:49,  1 user,  load average: 0.00, 0.01, 0.05
***.com    13:18:28 up 55 days, 13 min,  1 user,  load average: 0.08, 0.03, 0.05
\end{lstlisting}

To run the chef-client on all nodes:

\begin{lstlisting}[language=Bash,label=lst:my-server-cloud-knife-ssh2]
$ knife ssh 'name:*' 'sudo chef-client' -i ../keys/production.pem -x ubuntu
...
\end{lstlisting}

To run the chef-client on all nodes, which name begin from <<second>> string:

\begin{lstlisting}[language=Bash,label=lst:my-server-cloud-knife-ssh3]
$ knife ssh 'name:second*' 'sudo chef-client' -i ../keys/production.pem -x ubuntu
...
\end{lstlisting}

To upgrade all nodes:

\begin{lstlisting}[language=Bash,label=lst:my-server-cloud-knife-ssh4]
$ knife ssh 'name:*' 'sudo aptitude upgrade -y' -i ../keys/production.pem -x ubuntu
...
\end{lstlisting}

To get memory information from all nodes in staging environment:

\begin{lstlisting}[language=Bash,label=lst:my-server-cloud-knife-ssh5]
$ knife ssh "chef_environment:staging" "free -m" -i ../keys/production.pem -x ubuntu
***.com              total       used       free     shared    buffers     cached
***.com Mem:          1692       1182        509          0        181        491
***.com -/+ buffers/cache:        509       1183
***.com Swap:          895          6        889
...
\end{lstlisting}

\subsection{Chef-client cookbook}

\href{http://community.opscode.com/cookbooks/chef-client}{chef-client}