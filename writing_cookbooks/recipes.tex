\section{Recipes}

Any cookbook contains recipes. The default recipe inside cookbook have name <<default>>. Let's add our default recipe, which will install \href{http://git-scm.com/}{git}:

\begin{lstlisting}[label=lst:cookbook-recipes1,title=my-server-cloud/site-cookbooks/my\_cool\_app/recipes/default.rb]
#
# Cookbook Name:: my_cool_app
# Recipe:: default
#
# Copyright (C) 2014 Alexey Vasiliev
#
# MIT
#

package 'git'
\end{lstlisting}

As you can see, at the beginning of recipe we have comments about this recipe. Next we add resource <<package>> with argument <<git>>. The <<package>> resource is used to manage packages on the system. For example, on Debian or Ubuntu resource <<package>> will use <<apt-get>> command to install git on system.

Now you should add <<my\_cool\_app>> into run-list to use this cookbook:

\begin{lstlisting}[label=lst:cookbook-recipes2,title=my-server-cloud/site-cookbooks/nodes/second.example.com.json]
{
  "name": "second.example.com",
  "json_class": "Chef::Node",
  "chef_type": "node",
  "chef_environment": "development",
  "normal": {
    "fqdn": "10.33.33.35"
  },
  "default": {},
  "override": {},
  "run_list": [
    "role[chef-client]",
    "role[nginx]",
    "my_cool_app"
  ]
}
\end{lstlisting}

If you using Chef Server, don't forget upload this cookbook and update node on Chef Server by knife.

\begin{lstlisting}[language=Bash,label=lst:cookbook-recipes3]
$ knife cookbook upload my_cool_app
Uploading my_cool_app    [0.1.0]
Uploaded 1 cookbook.
$ knife node from file nodes/second.example.com.json
Updated Node second.example.com!
// on real environment you will execute "knife ssh 'name:second.example.com' 'sudo chef-client' -i ../keys/production.pem -x ubuntu"
$ vagrant provision chef_second_client
[2014-01-21T18:21:55+00:00] INFO: Chef Run complete in 26.935610739 seconds
[2014-01-21T18:21:55+00:00] INFO: Running report handler
\end{lstlisting}

Let's install also \href{http://en.wikipedia.org/wiki/Network\_Time\_Protocol}{ntp} package in the same recipe. Because we have in recipe Ruby syntax, we can little \href{http://ru.wikipedia.org/wiki/Dont\_repeat\_yourself}{DRY} our code:

\begin{lstlisting}[label=lst:cookbook-recipes4,title=my-server-cloud/site-cookbooks/my\_cool\_app/recipes/default.rb]
%w(git ntp).each do |pack|
  package pack
end
\end{lstlisting}

Again upload cookbook and run chef-client:

\begin{lstlisting}[language=Bash,label=lst:cookbook-recipes5]
$ knife cookbook upload my_cool_app
Uploading my_cool_app    [0.1.0]
Uploaded 1 cookbook.
// on real environment you will execute "knife ssh 'name:second.example.com' 'sudo chef-client' -i ../keys/production.pem -x ubuntu"
$ vagrant provision chef_second_client
[2014-01-21T18:21:55+00:00] INFO: Chef Run complete in 26.935610739 seconds
[2014-01-21T18:21:55+00:00] INFO: Running report handler
$ vagrant ssh chef_second_client
...
vagrant@precise64:~$ ps ax | grep ntp
 1115 ?        Ss     0:00 /usr/sbin/ntpd -p /var/run/ntpd.pid -g -u 103:108
13839 pts/2    S+     0:00 grep --color=auto ntp
vagrant@precise64:~$ git --version
git version 1.7.9.5
\end{lstlisting}

As you can see our simple cookbook is working.