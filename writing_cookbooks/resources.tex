\section{Resources and Providers}

As you read from previous chapter, Chef inside have resources (in example we used <<package>> resource). A resource defines the actions that can be taken, such as when a package should be installed, whether a service should be enabled or restarted, which groups, users, or groups of users should be created, where to put a collection of files, what the name of a new directory should be, and so on. During a chef-client run, each resource is identified and then associated with a provider. The provider then does the work to complete the action defined by the resource. Each resource is processed in the same order as they appear in a recipe. The chef-client ensures that the same actions are taken the same way everywhere and that actions produce the same result every time. A resource is implemented within a recipe using Ruby.

Let's look at the most necessary resources.

\subsection{Bash}

The bash resource is used to execute scripts using the Bash interpreter and includes all of the actions and attributes that are available to the execute resource. Example:

\begin{lstlisting}[label=lst:cookbook-resources-bash]
bash "install_something" do
  user "root"
  cwd "/tmp"
  code <<-EOH
  wget http://www.example.com/tarball.tar.gz
  tar -zxf tarball.tar.gz
  cd tarball
  ./configure
  make
  make install
  EOH
end
\end{lstlisting}