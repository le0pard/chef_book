\section{HWRPs}

When Chef first came out, there was no Light Weight Resource Provider (LWRP) syntax and any hardcore extension to Chef had to be written in Ruby. However, Chef team saw a need to be filled and created LWRP, making it easier to create your own Resources. The problem comes when LWRP cannot fulfill all of your needs. This means you need to fall back to writing pure ruby code. For lack of a better term, I'll call this method a HWRP, or Heavy Weight Resource Provider.

While writing a LWRP is meant to be simple and elegant, writing a HWRP is meant to be flexible. It gives you the full power of ruby in exchange for elegance.

http://tech.yipit.com/2013/05/09/advanced-chef-writing-heavy-weight-resource-providers-hwrp/

https://github.com/opscode-cookbooks/database
https://github.com/sethvargo-cookbooks/swap