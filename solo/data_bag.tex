\section{Defining data bags}
\label{sec:solo-data-bag}

A data bag is a global variable that is stored as JSON data. The contents of a data bag can vary, but they often include sensitive information (such as database passwords).

Knife is not working with Chef Solo data bags, but we can use \href{http://thbishop.com/knife-solo\_data\_bag/}{knife-solo\_data\_bag} rubygem. Just add this gem in Gemfile:

\begin{lstlisting}[label=lst:my-cloud-chef-databag1,title=my-cloud/Gemfile]
source "https://rubygems.org"

gem 'knife-solo'
gem 'knife-solo_data_bag'
gem 'berkshelf'
\end{lstlisting}

and run command \lstinline!bundle! to install it. After installation your knife should have new commands to work with Chef Solo:

\begin{lstlisting}[language=Bash,label=lst:my-cloud-chef-databag2]
$ knife --help | grep solo
  knife solo cook [USER@]HOSTNAME [JSON] (options)
  knife solo init DIRECTORY
  knife solo prepare [USER@]HOSTNAME [JSON] (options)
knife solo bootstrap [USER@]HOSTNAME [JSON] (options)
knife solo clean [USER@]HOSTNAME
knife solo cook [USER@]HOSTNAME [JSON] (options)
knife solo init DIRECTORY
knife solo prepare [USER@]HOSTNAME [JSON] (options)
knife solo data bag create BAG [ITEM] (options)
knife solo data bag edit BAG ITEM (options)
knife solo data bag list (options)
knife solo data bag show BAG [ITEM] (options)
  knife solo clean [USER@]HOSTNAME
\end{lstlisting}

For beginning, let's create a plain text data bag:

\begin{lstlisting}[language=Bash,label=lst:my-cloud-chef-databag3]
$ EDITOR=vim knife solo data bag create pass mysql
Created data_bag_item[mysql]
\end{lstlisting}

\lstinline!EDITOR! environment variable is used to set editor, which will open data bag file. I add in this JSON password for database and save it. Now we can see the result:

\begin{lstlisting}[language=Bash,label=lst:my-cloud-chef-databag4]
$ knife solo data bag show pass
mysql:
  id:       mysql
  password: secret
\end{lstlisting}

If you open data bag file, you will see this JSON:

\begin{lstlisting}[language=JSON,label=lst:my-cloud-chef-databag10,title=my-cloud/data\_bags/pass/mysql.json]
{
  "name":"data_bag_item_pass_mysql",
  "chef_type":"data_bag_item",
  "json_class":"Chef::DataBagItem",
  "data_bag":"pass",
  "raw_data":{
    "id":"mysql",
    "password":"secret"
  }
}
\end{lstlisting}

Let's consider a json structure:

\begin{itemize}
  \item \lstinline!name! - a unique name of data bag
  \item \lstinline!chef_type! - this should always be set to \lstinline!data_bag_item!
  \item \lstinline!json_class! - this should always be set to Chef::DataBagItem
  \item \lstinline!data_bag! - name of data bag
  \item \lstinline!raw_data! - values of data bag
\end{itemize}

The contents of a data bag can be encrypted using shared secret encryption. This allows a data bag to store confidential information (such as a database password) or to be managed in a source control system (without plain-text data appearing in revision history).

Encrypting a data bag requires a secret key. A secret key can be created in any number of ways. For example, OpenSSL can be used to generate a random number, which can then be used as the secret key:

\begin{lstlisting}[language=Bash,label=lst:my-cloud-chef-databag5]
$ openssl rand -base64 512 | tr -d '\r\n' > .chef/encrypted_data_bag_secret
\end{lstlisting}

The \lstinline!tr! command eliminates any trailing line feeds. Doing so avoids key corruption when transferring the file between platforms with different line endings.

A data bag can be encrypted using a Knife command similar to:

\begin{lstlisting}[language=Bash,label=lst:my-cloud-chef-databag6]
$ EDITOR=vim knife solo data bag create passwords mysql --secret-file .chef/encrypted_data_bag_secret
Created data_bag_item[mysql]
\end{lstlisting}

Editor is opened with such content:

\begin{lstlisting}[language=Bash,label=lst:my-cloud-chef-databag20]
{
  "id": "mysql"
}
\end{lstlisting}

And you can add your password and save it:

\begin{lstlisting}[language=Bash,label=lst:my-cloud-chef-databag21]
{
  "id": "mysql",
  "password": "secret"
}
\end{lstlisting}

As a result we obtain the encrypted data bag:

\begin{lstlisting}[language=Bash,label=lst:my-cloud-chef-databag7]
$ knife solo data bag show passwords mysql
id:       mysql
password:
  cipher:         aes-256-cbc
  encrypted_data: ++RR0s5f3rypFO3+SZj25px9QHTFq7AN854F3XZotnQ=

  iv:             fxvNqHFHHbCNKntW8bBJkg==

  version:        1
\end{lstlisting}

An encrypted data bag item can be decrypted with a Knife command similar to:

\begin{lstlisting}[language=Bash,label=lst:my-cloud-chef-databag8,title=my-cloud/Gemfile]
$ knife solo data bag show passwords mysql --secret-file .chef/encrypted_data_bag_secret
id:       mysql
password: secret
\end{lstlisting}

You can set \lstinline!encrypted_data_bag_secret! in knife.rb file:

\begin{lstlisting}[label=lst:my-cloud-chef-databag9,title=my-cloud/.chef/knife.rb]
cookbook_path    ["cookbooks", "site-cookbooks"]
node_path        "nodes"
role_path        "roles"
environment_path "environments"
data_bag_path    "data_bags"
encrypted_data_bag_secret ".chef/encrypted_data_bag_secret"

knife[:berkshelf_path] = "cookbooks"
\end{lstlisting}

and in this case no need to define \lstinline!secret-file! for knife data bag commands:

\begin{lstlisting}[language=Bash,label=lst:my-cloud-chef-databag11]
$ knife solo data bag show passwords mysql
id:       mysql
password: secret
\end{lstlisting}

For Vagrant you should set \lstinline!encrypted_data_bag_secret_key_path!:

\begin{lstlisting}[label=lst:my-cloud-chef-databag12,title=my-cloud/Vagrantfile]
...
Vagrant.configure(VAGRANTFILE_API_VERSION) do |config|
  ...
    chef.encrypted_data_bag_secret_key_path = Chef::Config[:encrypted_data_bag_secret]
  ...
end
\end{lstlisting}

The Recipe DSL provides access to data bags and data bag items with the following methods: \lstinline!data_bag('bag')!, where bag is the name of the data bag and \lstinline!data_bag_item('bag', 'item')!, where bag is the name of the data bag and item is the name of the data bag item. Examples:

\begin{lstlisting}[label=lst:my-cloud-chef-databag13]
data_bag("pass")
# => ["mysql"]
item = data_bag_item("pass", "mysql")
item["password"]
# => "secret"
\end{lstlisting}

A recipe can access encrypted data bag items as long as the recipe is running on a node that has access to the shared-key that is required to decrypt the data. A secret can be specified by using the Chef::EncryptedDataBagItem.load method. For example:

\begin{lstlisting}[label=lst:my-cloud-chef-databag14]
mysql_creds = Chef::EncryptedDataBagItem.load("passwords", "mysql", secret_key)
mysql_creds["password"]
# => "secret"
\end{lstlisting}

where \lstinline!secret_key! is the argument that specifies the location of the file that contains the encryption key. An encryption key can be configured so that the chef-client knows where to look using the \lstinline!Chef::Config[:encrypted_data_bag_secret]! method, which defaults to \lstinline!/etc/chef/encrypted_data_bag_secret!. When the default location is used, the argument that specifies the secret key file location is assumed to be the default and does not need to be explicitly specified in the recipe. For example:

\begin{lstlisting}[label=lst:my-cloud-chef-databag15]
mysql_creds = Chef::EncryptedDataBagItem.load("passwords", "mysql")
mysql_creds["password"]
# => "secret"
\end{lstlisting}
