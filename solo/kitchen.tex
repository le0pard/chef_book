\section{Creation of kitchen (chef-repo)}
\label{sec:solo-kitchen}

Working with Chef Solo starts with creating kitchen (chef-repo). The kitchen is located on a workstation (the location from which most users will do most of their work, ie. your computer) and should be synchronized with a version control system. To create a kitchen use knife-solo rubygems:

\begin{lstlisting}[language=Bash,label=lst:my-cloud-kitchen1,title=my-cloud]
$ cd my-cloud
$ knife solo init .
WARNING: No knife configuration file found
Creating kitchen...
Creating knife.rb in kitchen...
Creating cupboards...
Setting up Berkshelf...
$ ls -o
total 32
-rw-r--r--  1 leo    14 Dec 14 00:36 Berksfile
-rw-r--r--@ 1 leo    63 Dec 14 00:36 Gemfile
-rw-r--r--  1 leo  4427 Dec 14 00:36 Gemfile.lock
drwxr-xr-x  3 leo   102 Dec 14 00:36 cookbooks
drwxr-xr-x  3 leo   102 Dec 14 00:36 data_bags
drwxr-xr-x  3 leo   102 Dec 14 00:36 environments
drwxr-xr-x  3 leo   102 Dec 14 00:36 nodes
drwxr-xr-x  3 leo   102 Dec 14 00:36 roles
drwxr-xr-x  3 leo   102 Dec 14 00:36 site-cookbooks
\end{lstlisting}

Let's consider the directory structure:

\begin{itemize}
  \item \lstinline!.chef! - a hidden directory that is used to store .pem files and the knife.rb file
  \item \lstinline!cookbooks! - directory for Chef cookbooks. This directory will be used for vendor cookbooks, that will be installed with the help of berkshelf
  \item \lstinline!data_bags! - directory for Chef Data Bags
  \item \lstinline!environments! - directory for Chef environments
  \item \lstinline!nodes! - directory for Chef nodes
  \item \lstinline!roles! - directory for Chef roles
  \item \lstinline!site-cookbooks! - directory for your custom Chef cookbooks
  \item \lstinline!Berksfile! - file contains a list of sources identifying which cookbooks to retrieve and where to get them for berkshelf (like Gemfile for rubygems)
\end{itemize}

<<Cookbooks>> directory added into <<.gitignore>>, because it contains only vendor cookbooks. Vendor cookbook data will not be modified, so there is no reason to keep them in VCS (git, mercurial, etc).
