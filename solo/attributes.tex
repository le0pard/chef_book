\section{Attributes}
\label{sec:solo-attributes}

An attribute is a specific detail about a node. Attributes are used by the chef-client to understand:

\begin{itemize}
  \item The current state of the node
  \item What the state of the node was at the end of the previous chef-client run
  \item What the state of the node should be at the end of the current chef-client run
\end{itemize}

In our example we install and configure by chef apache 2 web server. In vendor cookbook <<apache2>> exists default attributes.

\begin{lstlisting}[label=lst:my-cloud-attributes1]
# General settings
default['apache']['listen_ports'] = ["80"]
default['apache']['contact'] = "ops@example.com"
default['apache']['timeout'] = 300
default['apache']['keepalive'] = "On"
default['apache']['keepaliverequests'] = 100
default['apache']['keepalivetimeout'] = 5
...
\end{lstlisting}

This attributes used, because we don't override its by environment, role or node. For example, we can add 443 port through our <<web>> role:

\begin{lstlisting}[language=JSON,label=lst:my-cloud-attributes2]
{
  "name": "web",
  "description": "The base role for systems that serve web server",
  "chef_type": "role",
  "json_class": "Chef::Role",
  "default_attributes": {
    "apache": {
      "listen_ports": ["80", "443"]
    }
  },
  "run_list": [
    "recipe[apache2]"
  ]
}
\end{lstlisting}

Also we can set \inline{keepalivetimeout} for apache in node attributes:

\begin{lstlisting}[language=JSON,label=lst:my-cloud-attributes3]
{
  "apache": {
    "keepalivetimeout": 30
  },
  "run_list": [
    "role[web]"
  ]
}
\end{lstlisting}

If we will set \inline{listen_ports} in node attribute, when this will override role \inline{listen_ports} attribute:

\begin{lstlisting}[language=JSON,label=lst:my-cloud-attributes3]
{
  "apache": {
    "keepalivetimeout": 30,
    "listen_ports": ["80", "443", "8080"]
  },
  "run_list": [
    "role[web]"
  ]
}
\end{lstlisting}

But role can override attributes, that should be applied to all nodes, even if the node already has a value for an attribute. This is useful for ensuring that certain attributes always have specific values. All this attributes should be written in \inline{override_attributes} key (instead of \inline{default_attributes}):

\begin{lstlisting}[language=JSON,label=lst:my-cloud-attributes2]
{
  "name": "web",
  "description": "The base role for systems that serve web server",
  "chef_type": "role",
  "json_class": "Chef::Role",
  "override_attributes": {
    "apache": {
      "listen_ports": ["80", "443"]
    }
  },
  "run_list": [
    "recipe[apache2]"
  ]
}
\end{lstlisting}

More info about attributes you can find on \href{http://docs.opscode.com/chef_overview_attributes.html}{wiki page}.